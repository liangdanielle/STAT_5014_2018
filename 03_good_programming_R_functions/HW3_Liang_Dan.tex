\documentclass[]{article}
\usepackage{lmodern}
\usepackage{amssymb,amsmath}
\usepackage{ifxetex,ifluatex}
\usepackage{fixltx2e} % provides \textsubscript
\ifnum 0\ifxetex 1\fi\ifluatex 1\fi=0 % if pdftex
  \usepackage[T1]{fontenc}
  \usepackage[utf8]{inputenc}
\else % if luatex or xelatex
  \ifxetex
    \usepackage{mathspec}
  \else
    \usepackage{fontspec}
  \fi
  \defaultfontfeatures{Ligatures=TeX,Scale=MatchLowercase}
\fi
% use upquote if available, for straight quotes in verbatim environments
\IfFileExists{upquote.sty}{\usepackage{upquote}}{}
% use microtype if available
\IfFileExists{microtype.sty}{%
\usepackage{microtype}
\UseMicrotypeSet[protrusion]{basicmath} % disable protrusion for tt fonts
}{}
\usepackage[margin=1in]{geometry}
\usepackage{hyperref}
\hypersetup{unicode=true,
            pdftitle={HW3\_Liang\_Dan},
            pdfauthor={Dan Liang},
            pdfborder={0 0 0},
            breaklinks=true}
\urlstyle{same}  % don't use monospace font for urls
\usepackage{color}
\usepackage{fancyvrb}
\newcommand{\VerbBar}{|}
\newcommand{\VERB}{\Verb[commandchars=\\\{\}]}
\DefineVerbatimEnvironment{Highlighting}{Verbatim}{commandchars=\\\{\}}
% Add ',fontsize=\small' for more characters per line
\usepackage{framed}
\definecolor{shadecolor}{RGB}{248,248,248}
\newenvironment{Shaded}{\begin{snugshade}}{\end{snugshade}}
\newcommand{\KeywordTok}[1]{\textcolor[rgb]{0.13,0.29,0.53}{\textbf{#1}}}
\newcommand{\DataTypeTok}[1]{\textcolor[rgb]{0.13,0.29,0.53}{#1}}
\newcommand{\DecValTok}[1]{\textcolor[rgb]{0.00,0.00,0.81}{#1}}
\newcommand{\BaseNTok}[1]{\textcolor[rgb]{0.00,0.00,0.81}{#1}}
\newcommand{\FloatTok}[1]{\textcolor[rgb]{0.00,0.00,0.81}{#1}}
\newcommand{\ConstantTok}[1]{\textcolor[rgb]{0.00,0.00,0.00}{#1}}
\newcommand{\CharTok}[1]{\textcolor[rgb]{0.31,0.60,0.02}{#1}}
\newcommand{\SpecialCharTok}[1]{\textcolor[rgb]{0.00,0.00,0.00}{#1}}
\newcommand{\StringTok}[1]{\textcolor[rgb]{0.31,0.60,0.02}{#1}}
\newcommand{\VerbatimStringTok}[1]{\textcolor[rgb]{0.31,0.60,0.02}{#1}}
\newcommand{\SpecialStringTok}[1]{\textcolor[rgb]{0.31,0.60,0.02}{#1}}
\newcommand{\ImportTok}[1]{#1}
\newcommand{\CommentTok}[1]{\textcolor[rgb]{0.56,0.35,0.01}{\textit{#1}}}
\newcommand{\DocumentationTok}[1]{\textcolor[rgb]{0.56,0.35,0.01}{\textbf{\textit{#1}}}}
\newcommand{\AnnotationTok}[1]{\textcolor[rgb]{0.56,0.35,0.01}{\textbf{\textit{#1}}}}
\newcommand{\CommentVarTok}[1]{\textcolor[rgb]{0.56,0.35,0.01}{\textbf{\textit{#1}}}}
\newcommand{\OtherTok}[1]{\textcolor[rgb]{0.56,0.35,0.01}{#1}}
\newcommand{\FunctionTok}[1]{\textcolor[rgb]{0.00,0.00,0.00}{#1}}
\newcommand{\VariableTok}[1]{\textcolor[rgb]{0.00,0.00,0.00}{#1}}
\newcommand{\ControlFlowTok}[1]{\textcolor[rgb]{0.13,0.29,0.53}{\textbf{#1}}}
\newcommand{\OperatorTok}[1]{\textcolor[rgb]{0.81,0.36,0.00}{\textbf{#1}}}
\newcommand{\BuiltInTok}[1]{#1}
\newcommand{\ExtensionTok}[1]{#1}
\newcommand{\PreprocessorTok}[1]{\textcolor[rgb]{0.56,0.35,0.01}{\textit{#1}}}
\newcommand{\AttributeTok}[1]{\textcolor[rgb]{0.77,0.63,0.00}{#1}}
\newcommand{\RegionMarkerTok}[1]{#1}
\newcommand{\InformationTok}[1]{\textcolor[rgb]{0.56,0.35,0.01}{\textbf{\textit{#1}}}}
\newcommand{\WarningTok}[1]{\textcolor[rgb]{0.56,0.35,0.01}{\textbf{\textit{#1}}}}
\newcommand{\AlertTok}[1]{\textcolor[rgb]{0.94,0.16,0.16}{#1}}
\newcommand{\ErrorTok}[1]{\textcolor[rgb]{0.64,0.00,0.00}{\textbf{#1}}}
\newcommand{\NormalTok}[1]{#1}
\usepackage{longtable,booktabs}
\usepackage{graphicx,grffile}
\makeatletter
\def\maxwidth{\ifdim\Gin@nat@width>\linewidth\linewidth\else\Gin@nat@width\fi}
\def\maxheight{\ifdim\Gin@nat@height>\textheight\textheight\else\Gin@nat@height\fi}
\makeatother
% Scale images if necessary, so that they will not overflow the page
% margins by default, and it is still possible to overwrite the defaults
% using explicit options in \includegraphics[width, height, ...]{}
\setkeys{Gin}{width=\maxwidth,height=\maxheight,keepaspectratio}
\IfFileExists{parskip.sty}{%
\usepackage{parskip}
}{% else
\setlength{\parindent}{0pt}
\setlength{\parskip}{6pt plus 2pt minus 1pt}
}
\setlength{\emergencystretch}{3em}  % prevent overfull lines
\providecommand{\tightlist}{%
  \setlength{\itemsep}{0pt}\setlength{\parskip}{0pt}}
\setcounter{secnumdepth}{0}
% Redefines (sub)paragraphs to behave more like sections
\ifx\paragraph\undefined\else
\let\oldparagraph\paragraph
\renewcommand{\paragraph}[1]{\oldparagraph{#1}\mbox{}}
\fi
\ifx\subparagraph\undefined\else
\let\oldsubparagraph\subparagraph
\renewcommand{\subparagraph}[1]{\oldsubparagraph{#1}\mbox{}}
\fi

%%% Use protect on footnotes to avoid problems with footnotes in titles
\let\rmarkdownfootnote\footnote%
\def\footnote{\protect\rmarkdownfootnote}

%%% Change title format to be more compact
\usepackage{titling}

% Create subtitle command for use in maketitle
\newcommand{\subtitle}[1]{
  \posttitle{
    \begin{center}\large#1\end{center}
    }
}

\setlength{\droptitle}{-2em}

  \title{HW3\_Liang\_Dan}
    \pretitle{\vspace{\droptitle}\centering\huge}
  \posttitle{\par}
    \author{Dan Liang}
    \preauthor{\centering\large\emph}
  \postauthor{\par}
      \predate{\centering\large\emph}
  \postdate{\par}
    \date{9/16/2018}


\begin{document}
\maketitle

\section{Problem 4}\label{problem-4}

The guides introduces good coding style, which includes readable, can be
modified, shared easily. I think a good coding style starts with a clear
logic of program, and lies in the uniformity of code. And good coding
style needs practice.

\section{Program 5}\label{program-5}

The code above gave back many suggestions. Some of them suggests that
the code needs to be clearer.

\section{Program 6}\label{program-6}

\begin{verbatim}
## -- Attaching packages ------------------------------------------------------------------- tidyverse 1.2.1 --
\end{verbatim}

\begin{verbatim}
## √ ggplot2 3.0.0     √ purrr   0.2.5
## √ tibble  1.4.2     √ dplyr   0.7.6
## √ tidyr   0.8.1     √ stringr 1.3.1
## √ readr   1.1.1     √ forcats 0.3.0
\end{verbatim}

\begin{verbatim}
## -- Conflicts ---------------------------------------------------------------------- tidyverse_conflicts() --
## x dplyr::filter() masks stats::filter()
## x dplyr::lag()    masks stats::lag()
\end{verbatim}

\begin{verbatim}
## 
## Please cite as:
\end{verbatim}

\begin{verbatim}
##  Hlavac, Marek (2018). stargazer: Well-Formatted Regression and Summary Statistics Tables.
\end{verbatim}

\begin{verbatim}
##  R package version 5.2.2. https://CRAN.R-project.org/package=stargazer
\end{verbatim}

\begin{verbatim}
## 
## Attaching package: 'data.table'
\end{verbatim}

\begin{verbatim}
## The following objects are masked from 'package:dplyr':
## 
##     between, first, last
\end{verbatim}

\begin{verbatim}
## The following object is masked from 'package:purrr':
## 
##     transpose
\end{verbatim}

\begin{verbatim}
## 
## Attaching package: 'lubridate'
\end{verbatim}

\begin{verbatim}
## The following objects are masked from 'package:data.table':
## 
##     hour, isoweek, mday, minute, month, quarter, second, wday,
##     week, yday, year
\end{verbatim}

\begin{verbatim}
## The following object is masked from 'package:base':
## 
##     date
\end{verbatim}

\begin{longtable}[]{@{}lrrrrrr@{}}
\caption{Data set Summary}\tabularnewline
\toprule
& Observer & Mean1 & Std1 & Mean2 & Std2 & Cov\tabularnewline
\midrule
\endfirsthead
\toprule
& Observer & Mean1 & Std1 & Mean2 & Std2 & Cov\tabularnewline
\midrule
\endhead
i & 1 & 54.26610 & 16.76983 & 47.83472 & 26.93974 &
-0.0641284\tabularnewline
i.1 & 2 & 54.26873 & 16.76924 & 47.83082 & 26.93573 &
-0.0685864\tabularnewline
i.2 & 3 & 54.26732 & 16.76001 & 47.83772 & 26.93004 &
-0.0683434\tabularnewline
i.3 & 4 & 54.26327 & 16.76514 & 47.83225 & 26.93540 &
-0.0644719\tabularnewline
i.4 & 5 & 54.26030 & 16.76774 & 47.83983 & 26.93019 &
-0.0603414\tabularnewline
i.5 & 6 & 54.26144 & 16.76590 & 47.83025 & 26.93988 &
-0.0617148\tabularnewline
i.6 & 7 & 54.26881 & 16.76670 & 47.83545 & 26.94000 &
-0.0685042\tabularnewline
i.7 & 8 & 54.26785 & 16.76676 & 47.83590 & 26.93610 &
-0.0689797\tabularnewline
i.8 & 9 & 54.26588 & 16.76885 & 47.83150 & 26.93861 &
-0.0686092\tabularnewline
i.9 & 10 & 54.26734 & 16.76896 & 47.83955 & 26.93027 &
-0.0629611\tabularnewline
i.10 & 11 & 54.26993 & 16.76996 & 47.83699 & 26.93768 &
-0.0694456\tabularnewline
i.11 & 12 & 54.26692 & 16.77000 & 47.83160 & 26.93790 &
-0.0665752\tabularnewline
i.12 & 13 & 54.26015 & 16.76996 & 47.83972 & 26.93000 &
-0.0655833\tabularnewline
\bottomrule
\end{longtable}

\includegraphics{HW3_Liang_Dan_files/figure-latex/unnamed-chunk-1-1.pdf}
\includegraphics{HW3_Liang_Dan_files/figure-latex/unnamed-chunk-1-2.pdf}

\begin{verbatim}
## Loading required package: sm
\end{verbatim}

\begin{verbatim}
## Package 'sm', version 2.2-5.4: type help(sm) for summary information
\end{verbatim}

\includegraphics{HW3_Liang_Dan_files/figure-latex/unnamed-chunk-1-3.pdf}
\includegraphics{HW3_Liang_Dan_files/figure-latex/unnamed-chunk-1-4.pdf}

\section{Problem 7}\label{problem-7}

\begin{Shaded}
\begin{Highlighting}[]
\CommentTok{# url<-"http://www2.isye.gatech.edu/~jeffwu/wuhamadabook/data/BloodPressure.dat"}
\CommentTok{# the url link can not be assessed, data set can not be found}
\CommentTok{# BloodPressure<-read.table(url, header=T, skip=1, fill=T, stringsAsFactors = F)}
\CommentTok{# colnames(BloodPressure)<-c("Day",paste("Device",1:3,sep="_"),"Day2",paste("Doctor",1:3,sep="_"))}
\CommentTok{# BloodPressurecleaned <- select(BloodPressure,-Day2) %>%gather(Approach,Value,Device_1:Doctor_3) %>% arrange(Day)}

\CommentTok{# knitr::kable(summary(BloodPressurecleaned), caption="Blood Pressure summary")}
\end{Highlighting}
\end{Shaded}

Data link is broken, can not be assessed, the data cannot be find, the
code has listed above.

\section{Problem 8}\label{problem-8}

\includegraphics{HW3_Liang_Dan_files/figure-latex/unnamed-chunk-3-1.pdf}


\end{document}
